\documentclass{article}
\usepackage[top=12mm,bottom=12mm,left=30mm,right=30mm,head=12mm,includeheadfoot]{geometry}

\begin{document}

\begin{center}{\Large \textbf{
    Simulation-based inference with normalizing flows:\\
    Progress reports
    }}
\end{center}

\section*{What we did in week 1}

% What the general goal of our group was to achieve this week
\paragraph{General:}
Our goal was to familiarize ourselves with the theoretical background of our topic by reading relevant papers and to complete the first milestone--using SBI in a linear regression problem.
% What we did individually, we dont have to include alot of detail
\paragraph{David Littel:}
Mainly read the introductory theory provided and got familiar with SBI using Sven's linear model implementation.

\paragraph{Sokratis Parmagkos:}
Toyed with the sbi library on a linear regression problem, and reviewed some theory from the suggested sources.

\paragraph{Katerina Theodoridou:}
Focused on studying the research papers for SBI and NPE that were provided. Additionally, ran an example implementation of the linear regression problem using the sbi library.

\paragraph{Sven van Zijl:}
Got familiar with the sbi library by applying it on a linear model and I read the suggested papers.

\section*{What we will do in week 2}
We should study the theory more deeply to fill in the gaps in our understanding, while clarifying any questions from week 1 during our meetings with the mentors. Furthermore, complete the second milestone--comparing performance across various dimensions and selecting the simulator that will be our main focus in this project.

\section*{What we did in week 2}

% What the general goal of our group was to achieve this week
\paragraph{General:}
Our goal was to focus more on our assigned tasks individually. Specifically, we went into more detail with both the linear regression and the gravitational model, trying to determine which changes to the settings of the sbi causes which changes to the results.

% What we did individually, we dont have to include alot of detail
\paragraph{David Littel:}
Looked into gravitational wave data archives, worked on implementing a simpler sinus model for the GW simulator and simplifying the gravitational wave model.

\paragraph{Sokratis Parmagkos:}
Completed the linear regression code and explored extreme cases on the problem, in order to see where sbi fails.
Tried to optimize the code, so it produces consistent results even when the random number generator seed is not specified. Reviewed Sven's code on the simulator of gravitational waves.

\paragraph{Katerina Theodoridou:}
Studied the SBI and NPE papers in more detail, created an analysis document of the linear regression problem. Also, added the paiplots in the linear regression code. 

\paragraph{Sven van Zijl:}
I created the simulator of the gravitational waves, implemented normalization, made a custom embedding and neural network for the interference, and I played around with simpler models for the gravitational waves trying to get a better result.


\section*{What we will do in week 3}
We will finalize and rehearse the group presentation, and make final adjustments to the code to improve visualizations and enhance the overall results.


\section*{What we did in week 3}
% What the general goal of our group was to achieve this week
\paragraph{General:}
The goal for this week was to complete the Gravitational waves inference problem and to prepare for the presentation.

% What we did individually, we dont have to include alot of detail
\paragraph{David Littel:}
Helped Sven implement the Gaussian noise into the simple GW model and added scaling by frequency and added plots showing the observed data and samples from the trained model to GW code.

\paragraph{Sokratis Parmagkos:}
I summed up the theory and made some slides for the presentation. I reviewed the final code corrections implemented from Sven and David on the GWs problem. Finally, I prepared for the presentation (rehearsed so I don't exceed time).

\paragraph{Katerina Theodoridou:}
I helped with the format of the presentation, as well as the plotting of the linear regression problem and code. Furthermore, I prepared for the presentation and revised the appropriate papers. 

\paragraph{Sven van Zijl:}
Extended the interference from 2 parameters $(A, t_c)$ to 3 parameters $(d_L, M = m_1 + m_2, t_c)$ in order to get a more interpretable result to present. 


\end{document}
